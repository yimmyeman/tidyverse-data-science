% Options for packages loaded elsewhere
\PassOptionsToPackage{unicode}{hyperref}
\PassOptionsToPackage{hyphens}{url}
%
\documentclass[
]{article}
\usepackage{amsmath,amssymb}
\usepackage{lmodern}
\usepackage{iftex}
\ifPDFTeX
  \usepackage[T1]{fontenc}
  \usepackage[utf8]{inputenc}
  \usepackage{textcomp} % provide euro and other symbols
\else % if luatex or xetex
  \usepackage{unicode-math}
  \defaultfontfeatures{Scale=MatchLowercase}
  \defaultfontfeatures[\rmfamily]{Ligatures=TeX,Scale=1}
\fi
% Use upquote if available, for straight quotes in verbatim environments
\IfFileExists{upquote.sty}{\usepackage{upquote}}{}
\IfFileExists{microtype.sty}{% use microtype if available
  \usepackage[]{microtype}
  \UseMicrotypeSet[protrusion]{basicmath} % disable protrusion for tt fonts
}{}
\makeatletter
\@ifundefined{KOMAClassName}{% if non-KOMA class
  \IfFileExists{parskip.sty}{%
    \usepackage{parskip}
  }{% else
    \setlength{\parindent}{0pt}
    \setlength{\parskip}{6pt plus 2pt minus 1pt}}
}{% if KOMA class
  \KOMAoptions{parskip=half}}
\makeatother
\usepackage{xcolor}
\usepackage[margin=1in]{geometry}
\usepackage{color}
\usepackage{fancyvrb}
\newcommand{\VerbBar}{|}
\newcommand{\VERB}{\Verb[commandchars=\\\{\}]}
\DefineVerbatimEnvironment{Highlighting}{Verbatim}{commandchars=\\\{\}}
% Add ',fontsize=\small' for more characters per line
\usepackage{framed}
\definecolor{shadecolor}{RGB}{248,248,248}
\newenvironment{Shaded}{\begin{snugshade}}{\end{snugshade}}
\newcommand{\AlertTok}[1]{\textcolor[rgb]{0.94,0.16,0.16}{#1}}
\newcommand{\AnnotationTok}[1]{\textcolor[rgb]{0.56,0.35,0.01}{\textbf{\textit{#1}}}}
\newcommand{\AttributeTok}[1]{\textcolor[rgb]{0.77,0.63,0.00}{#1}}
\newcommand{\BaseNTok}[1]{\textcolor[rgb]{0.00,0.00,0.81}{#1}}
\newcommand{\BuiltInTok}[1]{#1}
\newcommand{\CharTok}[1]{\textcolor[rgb]{0.31,0.60,0.02}{#1}}
\newcommand{\CommentTok}[1]{\textcolor[rgb]{0.56,0.35,0.01}{\textit{#1}}}
\newcommand{\CommentVarTok}[1]{\textcolor[rgb]{0.56,0.35,0.01}{\textbf{\textit{#1}}}}
\newcommand{\ConstantTok}[1]{\textcolor[rgb]{0.00,0.00,0.00}{#1}}
\newcommand{\ControlFlowTok}[1]{\textcolor[rgb]{0.13,0.29,0.53}{\textbf{#1}}}
\newcommand{\DataTypeTok}[1]{\textcolor[rgb]{0.13,0.29,0.53}{#1}}
\newcommand{\DecValTok}[1]{\textcolor[rgb]{0.00,0.00,0.81}{#1}}
\newcommand{\DocumentationTok}[1]{\textcolor[rgb]{0.56,0.35,0.01}{\textbf{\textit{#1}}}}
\newcommand{\ErrorTok}[1]{\textcolor[rgb]{0.64,0.00,0.00}{\textbf{#1}}}
\newcommand{\ExtensionTok}[1]{#1}
\newcommand{\FloatTok}[1]{\textcolor[rgb]{0.00,0.00,0.81}{#1}}
\newcommand{\FunctionTok}[1]{\textcolor[rgb]{0.00,0.00,0.00}{#1}}
\newcommand{\ImportTok}[1]{#1}
\newcommand{\InformationTok}[1]{\textcolor[rgb]{0.56,0.35,0.01}{\textbf{\textit{#1}}}}
\newcommand{\KeywordTok}[1]{\textcolor[rgb]{0.13,0.29,0.53}{\textbf{#1}}}
\newcommand{\NormalTok}[1]{#1}
\newcommand{\OperatorTok}[1]{\textcolor[rgb]{0.81,0.36,0.00}{\textbf{#1}}}
\newcommand{\OtherTok}[1]{\textcolor[rgb]{0.56,0.35,0.01}{#1}}
\newcommand{\PreprocessorTok}[1]{\textcolor[rgb]{0.56,0.35,0.01}{\textit{#1}}}
\newcommand{\RegionMarkerTok}[1]{#1}
\newcommand{\SpecialCharTok}[1]{\textcolor[rgb]{0.00,0.00,0.00}{#1}}
\newcommand{\SpecialStringTok}[1]{\textcolor[rgb]{0.31,0.60,0.02}{#1}}
\newcommand{\StringTok}[1]{\textcolor[rgb]{0.31,0.60,0.02}{#1}}
\newcommand{\VariableTok}[1]{\textcolor[rgb]{0.00,0.00,0.00}{#1}}
\newcommand{\VerbatimStringTok}[1]{\textcolor[rgb]{0.31,0.60,0.02}{#1}}
\newcommand{\WarningTok}[1]{\textcolor[rgb]{0.56,0.35,0.01}{\textbf{\textit{#1}}}}
\usepackage{graphicx}
\makeatletter
\def\maxwidth{\ifdim\Gin@nat@width>\linewidth\linewidth\else\Gin@nat@width\fi}
\def\maxheight{\ifdim\Gin@nat@height>\textheight\textheight\else\Gin@nat@height\fi}
\makeatother
% Scale images if necessary, so that they will not overflow the page
% margins by default, and it is still possible to overwrite the defaults
% using explicit options in \includegraphics[width, height, ...]{}
\setkeys{Gin}{width=\maxwidth,height=\maxheight,keepaspectratio}
% Set default figure placement to htbp
\makeatletter
\def\fps@figure{htbp}
\makeatother
\setlength{\emergencystretch}{3em} % prevent overfull lines
\providecommand{\tightlist}{%
  \setlength{\itemsep}{0pt}\setlength{\parskip}{0pt}}
\setcounter{secnumdepth}{-\maxdimen} % remove section numbering
\ifLuaTeX
  \usepackage{selnolig}  % disable illegal ligatures
\fi
\IfFileExists{bookmark.sty}{\usepackage{bookmark}}{\usepackage{hyperref}}
\IfFileExists{xurl.sty}{\usepackage{xurl}}{} % add URL line breaks if available
\urlstyle{same} % disable monospaced font for URLs
\hypersetup{
  pdftitle={Tarea15},
  pdfauthor={Yimmy Eman},
  hidelinks,
  pdfcreator={LaTeX via pandoc}}

\title{Tarea15}
\author{Yimmy Eman}
\date{2022-07-11}

\begin{document}
\maketitle

\hypertarget{pregunta-1}{%
\section{Pregunta 1}\label{pregunta-1}}

Indica qué función y parámetros usarías para leer ficheros separados con
``\textbar{}''

read\_delim(``path'', delim = ``\textbar{}'')

\hypertarget{pregunta-2}{%
\section{Pregunta 2}\label{pregunta-2}}

Además de file, skip y comment que hemos visto en el curso, ¿qué otros
argumentos tienen read\_csv y read\_tsv en común? Indica para qué sirve
cada uno de ellos.

col\_names = TRUE, col\_types = NULL, col\_select = NULL, id = NULL,
locale = default\_locale(), na = c(````,''NA''), quoted\_na = TRUE,
quote = ``"'', comment = ````, trim\_ws = TRUE, skip = 0, n\_max = Inf,
guess\_max = min(1000, n\_max), progress = show\_progress(),
name\_repair =''unique'', num\_threads = readr\_threads(),
show\_col\_types = should\_show\_types(), skip\_empty\_rows = TRUE, lazy
= should\_read\_lazy()\# Pregunta 3

Indica los argumentos más importantes de read\_fwf() R: col\_names,
col\_types, locale, na, quoted\_na, trim\_ws, n\_max, guess\_max,
progress

\hypertarget{pregunta-4}{%
\section{Pregunta 4}\label{pregunta-4}}

A veces un csv contiene necesariamente comas en los campos que son
strings. Para evitar problemas en la carga, suelen ir rodeadas de
comillas dobles '' o de comillas simples '. La convención de read\_csv()
es que asume que cualquier caracter vendrá rodeado por comillas dobles
'' y si lo queremos cambiar tenemos que usar la función read\_delim().
Indica qué argumentos tendríamos que especificar para poder leer el
texto del siguiente data frame correctamente

``x,y\n1,`a,b'\,''

\begin{Shaded}
\begin{Highlighting}[]
\NormalTok{data }\OtherTok{\textless{}{-}} \StringTok{"x,y}\SpecialCharTok{\textbackslash{}n}\StringTok{1,\textquotesingle{}a,b\textquotesingle{}"}
\FunctionTok{read\_delim}\NormalTok{(data, }\StringTok{","}\NormalTok{, }\AttributeTok{quote =} \StringTok{"\textquotesingle{}"}\NormalTok{)}
\end{Highlighting}
\end{Shaded}

\begin{verbatim}
## Rows: 1 Columns: 2
## -- Column specification --------------------------------------------------------
## Delimiter: ","
## chr (1): y
## dbl (1): x
## 
## i Use `spec()` to retrieve the full column specification for this data.
## i Specify the column types or set `show_col_types = FALSE` to quiet this message.
\end{verbatim}

\begin{verbatim}
## # A tibble: 1 x 2
##       x y    
##   <dbl> <chr>
## 1     1 a,b
\end{verbatim}

\begin{Shaded}
\begin{Highlighting}[]
\NormalTok{data}
\end{Highlighting}
\end{Shaded}

\begin{verbatim}
## [1] "x,y\n1,'a,b'"
\end{verbatim}

\hypertarget{pregunta-5}{%
\section{Pregunta 5}\label{pregunta-5}}

Indica qué está mal en la siguiente línea de lectura de CSV:
read\_csv(``x,y\n1,2,3\n4,5,6'')

\begin{Shaded}
\begin{Highlighting}[]
\FunctionTok{read\_csv}\NormalTok{(}\StringTok{"x,y}\SpecialCharTok{\textbackslash{}n}\StringTok{1,2,3}\SpecialCharTok{\textbackslash{}n}\StringTok{4,5,6"}\NormalTok{)}
\end{Highlighting}
\end{Shaded}

\begin{verbatim}
## Warning: One or more parsing issues, see `problems()` for details
\end{verbatim}

\begin{verbatim}
## Rows: 2 Columns: 2
## -- Column specification --------------------------------------------------------
## Delimiter: ","
## dbl (1): x
## 
## i Use `spec()` to retrieve the full column specification for this data.
## i Specify the column types or set `show_col_types = FALSE` to quiet this message.
\end{verbatim}

\begin{verbatim}
## # A tibble: 2 x 2
##       x     y
##   <dbl> <dbl>
## 1     1    23
## 2     4    56
\end{verbatim}

Sólo hay dos columnas, pero 3 en la entrada de datos.

\hypertarget{pregunta-6}{%
\section{Pregunta 6}\label{pregunta-6}}

Indica qué está mal en la siguiente línea de lectura de CSV:
read\_csv(``x,y,z\n1,2\n3,4,5,6'')

\begin{Shaded}
\begin{Highlighting}[]
\FunctionTok{read\_csv}\NormalTok{(}\StringTok{"x,y,z}\SpecialCharTok{\textbackslash{}n}\StringTok{1,2}\SpecialCharTok{\textbackslash{}n}\StringTok{3,4,5,6"}\NormalTok{)}
\end{Highlighting}
\end{Shaded}

\begin{verbatim}
## Warning: One or more parsing issues, see `problems()` for details
\end{verbatim}

\begin{verbatim}
## Rows: 2 Columns: 3
## -- Column specification --------------------------------------------------------
## Delimiter: ","
## dbl (2): x, y
## 
## i Use `spec()` to retrieve the full column specification for this data.
## i Specify the column types or set `show_col_types = FALSE` to quiet this message.
\end{verbatim}

\begin{verbatim}
## # A tibble: 2 x 3
##       x     y     z
##   <dbl> <dbl> <dbl>
## 1     1     2    NA
## 2     3     4    56
\end{verbatim}

Las filas de datos tienen menos o más columnas de información de la
cabecera. En la lectura, la primera filatendrá un NA mientras que la
última eliminará la última columna de información.

\hypertarget{pregunta-7}{%
\section{Pregunta 7}\label{pregunta-7}}

Indica qué está mal en la siguiente línea de lectura de CSV:
read\_csv(``x,y\n"1'')

\begin{Shaded}
\begin{Highlighting}[]
\FunctionTok{read\_csv}\NormalTok{(}\StringTok{"x,y}\SpecialCharTok{\textbackslash{}n\textbackslash{}"}\StringTok{1"}\NormalTok{)}
\end{Highlighting}
\end{Shaded}

\begin{verbatim}
## Rows: 0 Columns: 2
## -- Column specification --------------------------------------------------------
## Delimiter: ","
## chr (2): x, y
## 
## i Use `spec()` to retrieve the full column specification for this data.
## i Specify the column types or set `show_col_types = FALSE` to quiet this message.
\end{verbatim}

\begin{verbatim}
## # A tibble: 0 x 2
## # ... with 2 variables: x <chr>, y <chr>
\end{verbatim}

El caracter escapante está mal indicado.

\hypertarget{pregunta-8}{%
\section{Pregunta 8}\label{pregunta-8}}

Indica qué está mal en la siguiente línea de lectura de CSV:
read\_csv(``x,y\n1,2\na,b'')

\begin{Shaded}
\begin{Highlighting}[]
\FunctionTok{read\_csv}\NormalTok{(}\StringTok{"x,y}\SpecialCharTok{\textbackslash{}n}\StringTok{1,2}\SpecialCharTok{\textbackslash{}n}\StringTok{a,b"}\NormalTok{)}
\end{Highlighting}
\end{Shaded}

\begin{verbatim}
## Rows: 2 Columns: 2
## -- Column specification --------------------------------------------------------
## Delimiter: ","
## chr (2): x, y
## 
## i Use `spec()` to retrieve the full column specification for this data.
## i Specify the column types or set `show_col_types = FALSE` to quiet this message.
\end{verbatim}

\begin{verbatim}
## # A tibble: 2 x 2
##   x     y    
##   <chr> <chr>
## 1 1     2    
## 2 a     b
\end{verbatim}

Las columnas de datos no son homogéneas.

\hypertarget{pregunta-9}{%
\section{Pregunta 9}\label{pregunta-9}}

Indica qué está mal en la siguiente línea de lectura de CSV:
read\_csv(``x;y\n1;2'')

\begin{Shaded}
\begin{Highlighting}[]
\FunctionTok{read\_csv}\NormalTok{(}\StringTok{"x;y}\SpecialCharTok{\textbackslash{}n}\StringTok{1;2"}\NormalTok{)}
\end{Highlighting}
\end{Shaded}

\begin{verbatim}
## Rows: 1 Columns: 1
## -- Column specification --------------------------------------------------------
## Delimiter: ","
## chr (1): x;y
## 
## i Use `spec()` to retrieve the full column specification for this data.
## i Specify the column types or set `show_col_types = FALSE` to quiet this message.
\end{verbatim}

\begin{verbatim}
## # A tibble: 1 x 1
##   `x;y`
##   <chr>
## 1 1;2
\end{verbatim}

Se debe usar read\_csv2 por estar delimitado por ;

\end{document}
