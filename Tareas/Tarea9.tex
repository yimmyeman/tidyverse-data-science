% Options for packages loaded elsewhere
\PassOptionsToPackage{unicode}{hyperref}
\PassOptionsToPackage{hyphens}{url}
%
\documentclass[
]{article}
\usepackage{amsmath,amssymb}
\usepackage{lmodern}
\usepackage{iftex}
\ifPDFTeX
  \usepackage[T1]{fontenc}
  \usepackage[utf8]{inputenc}
  \usepackage{textcomp} % provide euro and other symbols
\else % if luatex or xetex
  \usepackage{unicode-math}
  \defaultfontfeatures{Scale=MatchLowercase}
  \defaultfontfeatures[\rmfamily]{Ligatures=TeX,Scale=1}
\fi
% Use upquote if available, for straight quotes in verbatim environments
\IfFileExists{upquote.sty}{\usepackage{upquote}}{}
\IfFileExists{microtype.sty}{% use microtype if available
  \usepackage[]{microtype}
  \UseMicrotypeSet[protrusion]{basicmath} % disable protrusion for tt fonts
}{}
\makeatletter
\@ifundefined{KOMAClassName}{% if non-KOMA class
  \IfFileExists{parskip.sty}{%
    \usepackage{parskip}
  }{% else
    \setlength{\parindent}{0pt}
    \setlength{\parskip}{6pt plus 2pt minus 1pt}}
}{% if KOMA class
  \KOMAoptions{parskip=half}}
\makeatother
\usepackage{xcolor}
\usepackage[margin=1in]{geometry}
\usepackage{color}
\usepackage{fancyvrb}
\newcommand{\VerbBar}{|}
\newcommand{\VERB}{\Verb[commandchars=\\\{\}]}
\DefineVerbatimEnvironment{Highlighting}{Verbatim}{commandchars=\\\{\}}
% Add ',fontsize=\small' for more characters per line
\usepackage{framed}
\definecolor{shadecolor}{RGB}{248,248,248}
\newenvironment{Shaded}{\begin{snugshade}}{\end{snugshade}}
\newcommand{\AlertTok}[1]{\textcolor[rgb]{0.94,0.16,0.16}{#1}}
\newcommand{\AnnotationTok}[1]{\textcolor[rgb]{0.56,0.35,0.01}{\textbf{\textit{#1}}}}
\newcommand{\AttributeTok}[1]{\textcolor[rgb]{0.77,0.63,0.00}{#1}}
\newcommand{\BaseNTok}[1]{\textcolor[rgb]{0.00,0.00,0.81}{#1}}
\newcommand{\BuiltInTok}[1]{#1}
\newcommand{\CharTok}[1]{\textcolor[rgb]{0.31,0.60,0.02}{#1}}
\newcommand{\CommentTok}[1]{\textcolor[rgb]{0.56,0.35,0.01}{\textit{#1}}}
\newcommand{\CommentVarTok}[1]{\textcolor[rgb]{0.56,0.35,0.01}{\textbf{\textit{#1}}}}
\newcommand{\ConstantTok}[1]{\textcolor[rgb]{0.00,0.00,0.00}{#1}}
\newcommand{\ControlFlowTok}[1]{\textcolor[rgb]{0.13,0.29,0.53}{\textbf{#1}}}
\newcommand{\DataTypeTok}[1]{\textcolor[rgb]{0.13,0.29,0.53}{#1}}
\newcommand{\DecValTok}[1]{\textcolor[rgb]{0.00,0.00,0.81}{#1}}
\newcommand{\DocumentationTok}[1]{\textcolor[rgb]{0.56,0.35,0.01}{\textbf{\textit{#1}}}}
\newcommand{\ErrorTok}[1]{\textcolor[rgb]{0.64,0.00,0.00}{\textbf{#1}}}
\newcommand{\ExtensionTok}[1]{#1}
\newcommand{\FloatTok}[1]{\textcolor[rgb]{0.00,0.00,0.81}{#1}}
\newcommand{\FunctionTok}[1]{\textcolor[rgb]{0.00,0.00,0.00}{#1}}
\newcommand{\ImportTok}[1]{#1}
\newcommand{\InformationTok}[1]{\textcolor[rgb]{0.56,0.35,0.01}{\textbf{\textit{#1}}}}
\newcommand{\KeywordTok}[1]{\textcolor[rgb]{0.13,0.29,0.53}{\textbf{#1}}}
\newcommand{\NormalTok}[1]{#1}
\newcommand{\OperatorTok}[1]{\textcolor[rgb]{0.81,0.36,0.00}{\textbf{#1}}}
\newcommand{\OtherTok}[1]{\textcolor[rgb]{0.56,0.35,0.01}{#1}}
\newcommand{\PreprocessorTok}[1]{\textcolor[rgb]{0.56,0.35,0.01}{\textit{#1}}}
\newcommand{\RegionMarkerTok}[1]{#1}
\newcommand{\SpecialCharTok}[1]{\textcolor[rgb]{0.00,0.00,0.00}{#1}}
\newcommand{\SpecialStringTok}[1]{\textcolor[rgb]{0.31,0.60,0.02}{#1}}
\newcommand{\StringTok}[1]{\textcolor[rgb]{0.31,0.60,0.02}{#1}}
\newcommand{\VariableTok}[1]{\textcolor[rgb]{0.00,0.00,0.00}{#1}}
\newcommand{\VerbatimStringTok}[1]{\textcolor[rgb]{0.31,0.60,0.02}{#1}}
\newcommand{\WarningTok}[1]{\textcolor[rgb]{0.56,0.35,0.01}{\textbf{\textit{#1}}}}
\usepackage{graphicx}
\makeatletter
\def\maxwidth{\ifdim\Gin@nat@width>\linewidth\linewidth\else\Gin@nat@width\fi}
\def\maxheight{\ifdim\Gin@nat@height>\textheight\textheight\else\Gin@nat@height\fi}
\makeatother
% Scale images if necessary, so that they will not overflow the page
% margins by default, and it is still possible to overwrite the defaults
% using explicit options in \includegraphics[width, height, ...]{}
\setkeys{Gin}{width=\maxwidth,height=\maxheight,keepaspectratio}
% Set default figure placement to htbp
\makeatletter
\def\fps@figure{htbp}
\makeatother
\setlength{\emergencystretch}{3em} % prevent overfull lines
\providecommand{\tightlist}{%
  \setlength{\itemsep}{0pt}\setlength{\parskip}{0pt}}
\setcounter{secnumdepth}{-\maxdimen} % remove section numbering
\ifLuaTeX
  \usepackage{selnolig}  % disable illegal ligatures
\fi
\IfFileExists{bookmark.sty}{\usepackage{bookmark}}{\usepackage{hyperref}}
\IfFileExists{xurl.sty}{\usepackage{xurl}}{} % add URL line breaks if available
\urlstyle{same} % disable monospaced font for URLs
\hypersetup{
  pdftitle={Tarea9},
  pdfauthor={Yimmy Eman},
  hidelinks,
  pdfcreator={LaTeX via pandoc}}

\title{Tarea9}
\author{Yimmy Eman}
\date{2022-07-08}

\begin{document}
\maketitle

\hypertarget{pregunta-1}{%
\section{Pregunta 1}\label{pregunta-1}}

Piensa cómo podrías usar la función arrange() para colocar todos los
valores NA al inicio. Pista: puedes la función is.na() en lugar de la
función desc() como argumento de arrange.

\begin{Shaded}
\begin{Highlighting}[]
\FunctionTok{arrange}\NormalTok{(flights, }\SpecialCharTok{!}\FunctionTok{is.na}\NormalTok{(flights))}
\end{Highlighting}
\end{Shaded}

\begin{verbatim}
## # A tibble: 336,776 x 19
##     year month   day dep_time sched_dep_time dep_delay arr_time sched_arr_time
##    <int> <int> <int>    <int>          <int>     <dbl>    <int>          <int>
##  1  2013     1     2       NA           1545        NA       NA           1910
##  2  2013     1     2       NA           1601        NA       NA           1735
##  3  2013     1     3       NA            857        NA       NA           1209
##  4  2013     1     3       NA            645        NA       NA            952
##  5  2013     1     4       NA            845        NA       NA           1015
##  6  2013     1     4       NA           1830        NA       NA           2044
##  7  2013     1     5       NA            840        NA       NA           1001
##  8  2013     1     7       NA            820        NA       NA            958
##  9  2013     1     8       NA           1645        NA       NA           1838
## 10  2013     1     9       NA            755        NA       NA           1012
## # ... with 336,766 more rows, and 11 more variables: arr_delay <dbl>,
## #   carrier <chr>, flight <int>, tailnum <chr>, origin <chr>, dest <chr>,
## #   air_time <dbl>, distance <dbl>, hour <dbl>, minute <dbl>, time_hour <dttm>
\end{verbatim}

\hypertarget{pregunta-2}{%
\section{Pregunta 2}\label{pregunta-2}}

Ordena los vuelos de flights para encontrar los vuelos más retrasados en
la salida. ¿Qué vuelos fueron los que salieron los primeros antes de lo
previsto?

\begin{Shaded}
\begin{Highlighting}[]
\FunctionTok{arrange}\NormalTok{(flights, }\FunctionTok{desc}\NormalTok{(dep\_delay))[}\DecValTok{1}\NormalTok{,] }\CommentTok{\# Más retrasado}
\end{Highlighting}
\end{Shaded}

\begin{verbatim}
## # A tibble: 1 x 19
##    year month   day dep_time sched_dep_time dep_delay arr_time sched_arr_time
##   <int> <int> <int>    <int>          <int>     <dbl>    <int>          <int>
## 1  2013     1     9      641            900      1301     1242           1530
## # ... with 11 more variables: arr_delay <dbl>, carrier <chr>, flight <int>,
## #   tailnum <chr>, origin <chr>, dest <chr>, air_time <dbl>, distance <dbl>,
## #   hour <dbl>, minute <dbl>, time_hour <dttm>
\end{verbatim}

\begin{Shaded}
\begin{Highlighting}[]
\FunctionTok{arrange}\NormalTok{(flights, dep\_delay)[}\DecValTok{1}\NormalTok{,] }\CommentTok{\# Menos retrasado}
\end{Highlighting}
\end{Shaded}

\begin{verbatim}
## # A tibble: 1 x 19
##    year month   day dep_time sched_dep_time dep_delay arr_time sched_arr_time
##   <int> <int> <int>    <int>          <int>     <dbl>    <int>          <int>
## 1  2013    12     7     2040           2123       -43       40           2352
## # ... with 11 more variables: arr_delay <dbl>, carrier <chr>, flight <int>,
## #   tailnum <chr>, origin <chr>, dest <chr>, air_time <dbl>, distance <dbl>,
## #   hour <dbl>, minute <dbl>, time_hour <dttm>
\end{verbatim}

\hypertarget{pregunta-3}{%
\section{Pregunta 3}\label{pregunta-3}}

Ordena los vuelos de flights para encontrar los vuelos más rápidos. Usa
el concepto de rapidez que consideres.

\begin{Shaded}
\begin{Highlighting}[]
\FunctionTok{arrange}\NormalTok{(flights, }\FunctionTok{desc}\NormalTok{(distance }\SpecialCharTok{/}\NormalTok{ air\_time))}
\end{Highlighting}
\end{Shaded}

\begin{verbatim}
## # A tibble: 336,776 x 19
##     year month   day dep_time sched_dep_time dep_delay arr_time sched_arr_time
##    <int> <int> <int>    <int>          <int>     <dbl>    <int>          <int>
##  1  2013     5    25     1709           1700         9     1923           1937
##  2  2013     7     2     1558           1513        45     1745           1719
##  3  2013     5    13     2040           2025        15     2225           2226
##  4  2013     3    23     1914           1910         4     2045           2043
##  5  2013     1    12     1559           1600        -1     1849           1917
##  6  2013    11    17      650            655        -5     1059           1150
##  7  2013     2    21     2355           2358        -3      412            438
##  8  2013    11    17      759            800        -1     1212           1255
##  9  2013    11    16     2003           1925        38       17             36
## 10  2013    11    16     2349           2359       -10      402            440
## # ... with 336,766 more rows, and 11 more variables: arr_delay <dbl>,
## #   carrier <chr>, flight <int>, tailnum <chr>, origin <chr>, dest <chr>,
## #   air_time <dbl>, distance <dbl>, hour <dbl>, minute <dbl>, time_hour <dttm>
\end{verbatim}

\hypertarget{pregunta-4}{%
\section{Pregunta 4}\label{pregunta-4}}

¿Qué vuelos tienen los trayectos más largos? Busca en Wikipedia qué dos
aeropuertos del dataset alojan los vuelos más largos.

Vuelos entre el JFK de Nueva York y el HNL, aeropuerto internacional de
Honolulu en Hawai

\begin{Shaded}
\begin{Highlighting}[]
\FunctionTok{filter}\NormalTok{(flights, origin }\SpecialCharTok{==} \StringTok{"JFK"}\NormalTok{, dest }\SpecialCharTok{==} \StringTok{"HNL"}\NormalTok{)}
\end{Highlighting}
\end{Shaded}

\begin{verbatim}
## # A tibble: 342 x 19
##     year month   day dep_time sched_dep_time dep_delay arr_time sched_arr_time
##    <int> <int> <int>    <int>          <int>     <dbl>    <int>          <int>
##  1  2013     1     1      857            900        -3     1516           1530
##  2  2013     1     2      909            900         9     1525           1530
##  3  2013     1     3      914            900        14     1504           1530
##  4  2013     1     4      900            900         0     1516           1530
##  5  2013     1     5      858            900        -2     1519           1530
##  6  2013     1     6     1019            900        79     1558           1530
##  7  2013     1     7     1042            900       102     1620           1530
##  8  2013     1     8      901            900         1     1504           1530
##  9  2013     1     9      641            900      1301     1242           1530
## 10  2013     1    10      859            900        -1     1449           1530
## # ... with 332 more rows, and 11 more variables: arr_delay <dbl>,
## #   carrier <chr>, flight <int>, tailnum <chr>, origin <chr>, dest <chr>,
## #   air_time <dbl>, distance <dbl>, hour <dbl>, minute <dbl>, time_hour <dttm>
\end{verbatim}

\hypertarget{pregunta-5}{%
\section{Pregunta 5}\label{pregunta-5}}

¿Qué vuelos tienen los trayectos más cortos? Busca en Wikipedia qué dos
aeropuertos del dataset alojan los vuelos más cortos

\end{document}
