% Options for packages loaded elsewhere
\PassOptionsToPackage{unicode}{hyperref}
\PassOptionsToPackage{hyphens}{url}
%
\documentclass[
]{article}
\usepackage{amsmath,amssymb}
\usepackage{lmodern}
\usepackage{iftex}
\ifPDFTeX
  \usepackage[T1]{fontenc}
  \usepackage[utf8]{inputenc}
  \usepackage{textcomp} % provide euro and other symbols
\else % if luatex or xetex
  \usepackage{unicode-math}
  \defaultfontfeatures{Scale=MatchLowercase}
  \defaultfontfeatures[\rmfamily]{Ligatures=TeX,Scale=1}
\fi
% Use upquote if available, for straight quotes in verbatim environments
\IfFileExists{upquote.sty}{\usepackage{upquote}}{}
\IfFileExists{microtype.sty}{% use microtype if available
  \usepackage[]{microtype}
  \UseMicrotypeSet[protrusion]{basicmath} % disable protrusion for tt fonts
}{}
\makeatletter
\@ifundefined{KOMAClassName}{% if non-KOMA class
  \IfFileExists{parskip.sty}{%
    \usepackage{parskip}
  }{% else
    \setlength{\parindent}{0pt}
    \setlength{\parskip}{6pt plus 2pt minus 1pt}}
}{% if KOMA class
  \KOMAoptions{parskip=half}}
\makeatother
\usepackage{xcolor}
\usepackage[margin=1in]{geometry}
\usepackage{color}
\usepackage{fancyvrb}
\newcommand{\VerbBar}{|}
\newcommand{\VERB}{\Verb[commandchars=\\\{\}]}
\DefineVerbatimEnvironment{Highlighting}{Verbatim}{commandchars=\\\{\}}
% Add ',fontsize=\small' for more characters per line
\usepackage{framed}
\definecolor{shadecolor}{RGB}{248,248,248}
\newenvironment{Shaded}{\begin{snugshade}}{\end{snugshade}}
\newcommand{\AlertTok}[1]{\textcolor[rgb]{0.94,0.16,0.16}{#1}}
\newcommand{\AnnotationTok}[1]{\textcolor[rgb]{0.56,0.35,0.01}{\textbf{\textit{#1}}}}
\newcommand{\AttributeTok}[1]{\textcolor[rgb]{0.77,0.63,0.00}{#1}}
\newcommand{\BaseNTok}[1]{\textcolor[rgb]{0.00,0.00,0.81}{#1}}
\newcommand{\BuiltInTok}[1]{#1}
\newcommand{\CharTok}[1]{\textcolor[rgb]{0.31,0.60,0.02}{#1}}
\newcommand{\CommentTok}[1]{\textcolor[rgb]{0.56,0.35,0.01}{\textit{#1}}}
\newcommand{\CommentVarTok}[1]{\textcolor[rgb]{0.56,0.35,0.01}{\textbf{\textit{#1}}}}
\newcommand{\ConstantTok}[1]{\textcolor[rgb]{0.00,0.00,0.00}{#1}}
\newcommand{\ControlFlowTok}[1]{\textcolor[rgb]{0.13,0.29,0.53}{\textbf{#1}}}
\newcommand{\DataTypeTok}[1]{\textcolor[rgb]{0.13,0.29,0.53}{#1}}
\newcommand{\DecValTok}[1]{\textcolor[rgb]{0.00,0.00,0.81}{#1}}
\newcommand{\DocumentationTok}[1]{\textcolor[rgb]{0.56,0.35,0.01}{\textbf{\textit{#1}}}}
\newcommand{\ErrorTok}[1]{\textcolor[rgb]{0.64,0.00,0.00}{\textbf{#1}}}
\newcommand{\ExtensionTok}[1]{#1}
\newcommand{\FloatTok}[1]{\textcolor[rgb]{0.00,0.00,0.81}{#1}}
\newcommand{\FunctionTok}[1]{\textcolor[rgb]{0.00,0.00,0.00}{#1}}
\newcommand{\ImportTok}[1]{#1}
\newcommand{\InformationTok}[1]{\textcolor[rgb]{0.56,0.35,0.01}{\textbf{\textit{#1}}}}
\newcommand{\KeywordTok}[1]{\textcolor[rgb]{0.13,0.29,0.53}{\textbf{#1}}}
\newcommand{\NormalTok}[1]{#1}
\newcommand{\OperatorTok}[1]{\textcolor[rgb]{0.81,0.36,0.00}{\textbf{#1}}}
\newcommand{\OtherTok}[1]{\textcolor[rgb]{0.56,0.35,0.01}{#1}}
\newcommand{\PreprocessorTok}[1]{\textcolor[rgb]{0.56,0.35,0.01}{\textit{#1}}}
\newcommand{\RegionMarkerTok}[1]{#1}
\newcommand{\SpecialCharTok}[1]{\textcolor[rgb]{0.00,0.00,0.00}{#1}}
\newcommand{\SpecialStringTok}[1]{\textcolor[rgb]{0.31,0.60,0.02}{#1}}
\newcommand{\StringTok}[1]{\textcolor[rgb]{0.31,0.60,0.02}{#1}}
\newcommand{\VariableTok}[1]{\textcolor[rgb]{0.00,0.00,0.00}{#1}}
\newcommand{\VerbatimStringTok}[1]{\textcolor[rgb]{0.31,0.60,0.02}{#1}}
\newcommand{\WarningTok}[1]{\textcolor[rgb]{0.56,0.35,0.01}{\textbf{\textit{#1}}}}
\usepackage{graphicx}
\makeatletter
\def\maxwidth{\ifdim\Gin@nat@width>\linewidth\linewidth\else\Gin@nat@width\fi}
\def\maxheight{\ifdim\Gin@nat@height>\textheight\textheight\else\Gin@nat@height\fi}
\makeatother
% Scale images if necessary, so that they will not overflow the page
% margins by default, and it is still possible to overwrite the defaults
% using explicit options in \includegraphics[width, height, ...]{}
\setkeys{Gin}{width=\maxwidth,height=\maxheight,keepaspectratio}
% Set default figure placement to htbp
\makeatletter
\def\fps@figure{htbp}
\makeatother
\setlength{\emergencystretch}{3em} % prevent overfull lines
\providecommand{\tightlist}{%
  \setlength{\itemsep}{0pt}\setlength{\parskip}{0pt}}
\setcounter{secnumdepth}{-\maxdimen} % remove section numbering
\ifLuaTeX
  \usepackage{selnolig}  % disable illegal ligatures
\fi
\IfFileExists{bookmark.sty}{\usepackage{bookmark}}{\usepackage{hyperref}}
\IfFileExists{xurl.sty}{\usepackage{xurl}}{} % add URL line breaks if available
\urlstyle{same} % disable monospaced font for URLs
\hypersetup{
  pdftitle={Tarea16},
  pdfauthor={Yimmy Eman},
  hidelinks,
  pdfcreator={LaTeX via pandoc}}

\title{Tarea16}
\author{Yimmy Eman}
\date{2022-07-11}

\begin{document}
\maketitle

\begin{Shaded}
\begin{Highlighting}[]
\FunctionTok{library}\NormalTok{(tidyverse)}
\end{Highlighting}
\end{Shaded}

\hypertarget{pregunta-1}{%
\section{Pregunta 1}\label{pregunta-1}}

Investiga la documentación para decir cuales son los argumentos más
importantes que trae la función locale()

\begin{itemize}
\item
  date\_names\\
  Character representations of day and month names. Either the language
  code as string (passed on to date\_names\_lang()) or an object created
  by date\_names().
\item
  date\_format, time\_format\\
  Default date and time formats.
\item
  decimal\_mark, grouping\_mark\\
  Symbols used to indicate the decimal place, and to chunk larger
  numbers. Decimal mark can only be ⁠,⁠ or ..
\item
  tz\\
  Default tz. This is used both for input (if the time zone isn't
  present in individual strings), and for output (to control the default
  display). The default is to use ``UTC'', a time zone that does not use
  daylight savings time (DST) and hence is typically most useful for
  data. The absence of time zones makes it approximately 50x faster to
  generate UTC times than any other time zone. Use ``\,'' to use the
  system default time zone, but beware that this will not be
  reproducible across systems. For a complete list of possible time
  zones, see OlsonNames(). Americans, note that ``EST'' is a Canadian
  time zone that does not have DST. It is not Eastern Standard Time.
  It's better to use ``US/Eastern'', ``US/Central'' etc.
\item
  encoding\\
  Default encoding. This only affects how the file is read - readr
  always converts the output to UTF-8.
\item
  asciify\\
  Should diacritics be stripped from date names and converted to ASCII?
  This is useful if you're dealing with ASCII data where the correct
  spellings have been lost. Requires the stringi package.
\end{itemize}

\hypertarget{pregunta-2}{%
\section{Pregunta 2}\label{pregunta-2}}

\begin{itemize}
\tightlist
\item
  Investiga qué ocurre si intentamos configurar a la vez el
  decimal\_mark y grouping\_mark con el mismo carácter.
\end{itemize}

\begin{Shaded}
\begin{Highlighting}[]
\CommentTok{\#parse\_number("1.000.000", locale = locale(grouping\_mark = ".", decimal\_mark = "."))}
\end{Highlighting}
\end{Shaded}

Error: \texttt{decimal\_mark} and \texttt{grouping\_mark} must be
different

\begin{itemize}
\tightlist
\item
  ¿Qué valor por defecto toma el grouping\_mark cuando configuramos el
  decimal\_mark al carácter de coma ``,''?
\end{itemize}

\begin{Shaded}
\begin{Highlighting}[]
\FunctionTok{parse\_number}\NormalTok{(}\StringTok{"1.000.000"}\NormalTok{, }\AttributeTok{locale =} \FunctionTok{locale}\NormalTok{(}\AttributeTok{grouping\_mark =} \StringTok{"."}\NormalTok{, }\AttributeTok{decimal\_mark =} \StringTok{","}\NormalTok{))}
\end{Highlighting}
\end{Shaded}

\begin{verbatim}
## [1] 1e+06
\end{verbatim}

\begin{Shaded}
\begin{Highlighting}[]
\CommentTok{\# Queda configurado al contrario }
\end{Highlighting}
\end{Shaded}

\begin{itemize}
\tightlist
\item
  ¿Qué valor por defecto toma el decimal\_mark cuando configuramos el
  grouping\_mark al carácter de punto ``.''?
\end{itemize}

\begin{Shaded}
\begin{Highlighting}[]
\CommentTok{\#parse\_number("1.000.000", locale = locale(grouping\_mark = ".", decimal\_mark = "."))}
\end{Highlighting}
\end{Shaded}

\hypertarget{pregunta-3}{%
\section{Pregunta 3}\label{pregunta-3}}

Investiga qué hace la opción del locale() cuando se utiliza junto al
date\_format y al time\_format . Crea un ejemplo que muestre cuando
puede sernos útil.

\begin{Shaded}
\begin{Highlighting}[]
\FunctionTok{locale}\NormalTok{(}\AttributeTok{date\_names =} \StringTok{"en"}\NormalTok{,}
       \AttributeTok{date\_format =} \StringTok{"\%AD"}\NormalTok{,}
       \AttributeTok{time\_format =} \StringTok{"\%AT"}\NormalTok{,}
       \AttributeTok{decimal\_mark =} \StringTok{"."}\NormalTok{,}
       \AttributeTok{grouping\_mark =} \StringTok{","}\NormalTok{,}
       \AttributeTok{tz =} \StringTok{"UTC"}\NormalTok{,}
       \AttributeTok{encoding =} \StringTok{"UTF{-}8"}\NormalTok{,}
       \AttributeTok{asciify =} \ConstantTok{FALSE}
\NormalTok{)}
\end{Highlighting}
\end{Shaded}

\begin{verbatim}
## <locale>
## Numbers:  123,456.78
## Formats:  %AD / %AT
## Timezone: UTC
## Encoding: UTF-8
## <date_names>
## Days:   Sunday (Sun), Monday (Mon), Tuesday (Tue), Wednesday (Wed), Thursday
##         (Thu), Friday (Fri), Saturday (Sat)
## Months: January (Jan), February (Feb), March (Mar), April (Apr), May (May),
##         June (Jun), July (Jul), August (Aug), September (Sep), October
##         (Oct), November (Nov), December (Dec)
## AM/PM:  AM/PM
\end{verbatim}

\hypertarget{pregunta-4}{%
\section{Pregunta 4}\label{pregunta-4}}

Crea un nuevo objeto locale que encapsule los ajustes más comunes de los
parámetros para la carga de los fichero con los que sueles trabajar.

\begin{Shaded}
\begin{Highlighting}[]
\FunctionTok{locale}\NormalTok{(}\AttributeTok{date\_names =} \StringTok{"es"}\NormalTok{,}
       \AttributeTok{date\_format =} \StringTok{"\%AD"}\NormalTok{,}
       \AttributeTok{time\_format =} \StringTok{"\%AT"}\NormalTok{,}
       \AttributeTok{decimal\_mark =} \StringTok{","}\NormalTok{,}
       \AttributeTok{grouping\_mark =} \StringTok{"."}\NormalTok{,}
       \AttributeTok{tz =} \StringTok{"UTC"}\NormalTok{,}
       \AttributeTok{encoding =} \StringTok{"UTF{-}8"}\NormalTok{,}
       \AttributeTok{asciify =} \ConstantTok{FALSE}
\NormalTok{)}
\end{Highlighting}
\end{Shaded}

\begin{verbatim}
## <locale>
## Numbers:  123.456,78
## Formats:  %AD / %AT
## Timezone: UTC
## Encoding: UTF-8
## <date_names>
## Days:   domingo (dom.), lunes (lun.), martes (mar.), miércoles (mié.), jueves
##         (jue.), viernes (vie.), sábado (sáb.)
## Months: enero (ene.), febrero (feb.), marzo (mar.), abril (abr.), mayo (may.),
##         junio (jun.), julio (jul.), agosto (ago.), septiembre (sept.),
##         octubre (oct.), noviembre (nov.), diciembre (dic.)
## AM/PM:  a. m./p. m.
\end{verbatim}

\hypertarget{pregunta-5}{%
\section{Pregunta 5}\label{pregunta-5}}

Investiga las diferencias entre read\_csv() y read\_csv2()?

read\_csv() y read\_tsv() son casos especiales del read\_delim() más
general. Son útiles para leer los tipos más comunes de datos de archivos
sin formato, valores separados por comas y valores separados por
tabulaciones, respectivamente. read\_csv2() utiliza ?;? para el
separador de campos y ?,? para el punto decimal. Este formato es común
en algunos países europeos.

\hypertarget{pregunta-6}{%
\section{Pregunta 6}\label{pregunta-6}}

Investiga las codificaciones que son más frecuentes en Europa y las más
comunes en Asia. Usa un poco de Google e iniciativa para investigar
acerca de este tema.

\hypertarget{pregunta-7}{%
\section{Pregunta 7}\label{pregunta-7}}

Genera el formato correcto de string que procesa cada una de las
siguientes fechas y horas:

\begin{Shaded}
\begin{Highlighting}[]
\NormalTok{v1 }\OtherTok{\textless{}{-}} \StringTok{"May 19, 2018"}
\FunctionTok{parse\_date}\NormalTok{(v1, }\StringTok{"\%b \%d, \%Y"}\NormalTok{)}
\end{Highlighting}
\end{Shaded}

\begin{verbatim}
## [1] "2018-05-19"
\end{verbatim}

\begin{Shaded}
\begin{Highlighting}[]
\NormalTok{v2 }\OtherTok{\textless{}{-}} \StringTok{"2018{-}May{-}08"}
\FunctionTok{parse\_date}\NormalTok{(v2, }\StringTok{"\%Y{-}\%b{-}\%d"}\NormalTok{)}
\end{Highlighting}
\end{Shaded}

\begin{verbatim}
## [1] "2018-05-08"
\end{verbatim}

\begin{Shaded}
\begin{Highlighting}[]
\NormalTok{v3 }\OtherTok{\textless{}{-}} \StringTok{"09{-}Jul{-}2013"}
\FunctionTok{parse\_date}\NormalTok{(v3, }\StringTok{"\%d{-}\%b{-}\%Y"}\NormalTok{)}
\end{Highlighting}
\end{Shaded}

\begin{verbatim}
## [1] "2013-07-09"
\end{verbatim}

\begin{Shaded}
\begin{Highlighting}[]
\NormalTok{v4 }\OtherTok{\textless{}{-}} \FunctionTok{c}\NormalTok{(}\StringTok{"January 19 (2019)"}\NormalTok{, }\StringTok{"May 1 (2015)"}\NormalTok{)}
\FunctionTok{parse\_date}\NormalTok{(v4, }\StringTok{"\%B \%d (\%Y)"}\NormalTok{)}
\end{Highlighting}
\end{Shaded}

\begin{verbatim}
## [1] "2019-01-19" "2015-05-01"
\end{verbatim}

\begin{Shaded}
\begin{Highlighting}[]
\NormalTok{v5 }\OtherTok{\textless{}{-}} \StringTok{"12/31/18"} \CommentTok{\# Dic 31, 2014}
\FunctionTok{parse\_date}\NormalTok{(v5, }\StringTok{"\%m/\%d/\%y"}\NormalTok{)}
\end{Highlighting}
\end{Shaded}

\begin{verbatim}
## [1] "2018-12-31"
\end{verbatim}

\begin{Shaded}
\begin{Highlighting}[]
\NormalTok{v6 }\OtherTok{\textless{}{-}} \StringTok{"1305"}
\FunctionTok{parse\_time}\NormalTok{(v6, }\AttributeTok{format =} \StringTok{"\%H\%M"}\NormalTok{)}
\end{Highlighting}
\end{Shaded}

\begin{verbatim}
## 13:05:00
\end{verbatim}

\begin{Shaded}
\begin{Highlighting}[]
\NormalTok{v7 }\OtherTok{\textless{}{-}} \StringTok{"12:05:11.15 PM"}
\FunctionTok{parse\_time}\NormalTok{(v7)}
\end{Highlighting}
\end{Shaded}

\begin{verbatim}
## 12:05:11
\end{verbatim}

\end{document}
