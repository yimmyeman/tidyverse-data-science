% Options for packages loaded elsewhere
\PassOptionsToPackage{unicode}{hyperref}
\PassOptionsToPackage{hyphens}{url}
%
\documentclass[
]{article}
\usepackage{amsmath,amssymb}
\usepackage{lmodern}
\usepackage{iftex}
\ifPDFTeX
  \usepackage[T1]{fontenc}
  \usepackage[utf8]{inputenc}
  \usepackage{textcomp} % provide euro and other symbols
\else % if luatex or xetex
  \usepackage{unicode-math}
  \defaultfontfeatures{Scale=MatchLowercase}
  \defaultfontfeatures[\rmfamily]{Ligatures=TeX,Scale=1}
\fi
% Use upquote if available, for straight quotes in verbatim environments
\IfFileExists{upquote.sty}{\usepackage{upquote}}{}
\IfFileExists{microtype.sty}{% use microtype if available
  \usepackage[]{microtype}
  \UseMicrotypeSet[protrusion]{basicmath} % disable protrusion for tt fonts
}{}
\makeatletter
\@ifundefined{KOMAClassName}{% if non-KOMA class
  \IfFileExists{parskip.sty}{%
    \usepackage{parskip}
  }{% else
    \setlength{\parindent}{0pt}
    \setlength{\parskip}{6pt plus 2pt minus 1pt}}
}{% if KOMA class
  \KOMAoptions{parskip=half}}
\makeatother
\usepackage{xcolor}
\usepackage[margin=1in]{geometry}
\usepackage{color}
\usepackage{fancyvrb}
\newcommand{\VerbBar}{|}
\newcommand{\VERB}{\Verb[commandchars=\\\{\}]}
\DefineVerbatimEnvironment{Highlighting}{Verbatim}{commandchars=\\\{\}}
% Add ',fontsize=\small' for more characters per line
\usepackage{framed}
\definecolor{shadecolor}{RGB}{248,248,248}
\newenvironment{Shaded}{\begin{snugshade}}{\end{snugshade}}
\newcommand{\AlertTok}[1]{\textcolor[rgb]{0.94,0.16,0.16}{#1}}
\newcommand{\AnnotationTok}[1]{\textcolor[rgb]{0.56,0.35,0.01}{\textbf{\textit{#1}}}}
\newcommand{\AttributeTok}[1]{\textcolor[rgb]{0.77,0.63,0.00}{#1}}
\newcommand{\BaseNTok}[1]{\textcolor[rgb]{0.00,0.00,0.81}{#1}}
\newcommand{\BuiltInTok}[1]{#1}
\newcommand{\CharTok}[1]{\textcolor[rgb]{0.31,0.60,0.02}{#1}}
\newcommand{\CommentTok}[1]{\textcolor[rgb]{0.56,0.35,0.01}{\textit{#1}}}
\newcommand{\CommentVarTok}[1]{\textcolor[rgb]{0.56,0.35,0.01}{\textbf{\textit{#1}}}}
\newcommand{\ConstantTok}[1]{\textcolor[rgb]{0.00,0.00,0.00}{#1}}
\newcommand{\ControlFlowTok}[1]{\textcolor[rgb]{0.13,0.29,0.53}{\textbf{#1}}}
\newcommand{\DataTypeTok}[1]{\textcolor[rgb]{0.13,0.29,0.53}{#1}}
\newcommand{\DecValTok}[1]{\textcolor[rgb]{0.00,0.00,0.81}{#1}}
\newcommand{\DocumentationTok}[1]{\textcolor[rgb]{0.56,0.35,0.01}{\textbf{\textit{#1}}}}
\newcommand{\ErrorTok}[1]{\textcolor[rgb]{0.64,0.00,0.00}{\textbf{#1}}}
\newcommand{\ExtensionTok}[1]{#1}
\newcommand{\FloatTok}[1]{\textcolor[rgb]{0.00,0.00,0.81}{#1}}
\newcommand{\FunctionTok}[1]{\textcolor[rgb]{0.00,0.00,0.00}{#1}}
\newcommand{\ImportTok}[1]{#1}
\newcommand{\InformationTok}[1]{\textcolor[rgb]{0.56,0.35,0.01}{\textbf{\textit{#1}}}}
\newcommand{\KeywordTok}[1]{\textcolor[rgb]{0.13,0.29,0.53}{\textbf{#1}}}
\newcommand{\NormalTok}[1]{#1}
\newcommand{\OperatorTok}[1]{\textcolor[rgb]{0.81,0.36,0.00}{\textbf{#1}}}
\newcommand{\OtherTok}[1]{\textcolor[rgb]{0.56,0.35,0.01}{#1}}
\newcommand{\PreprocessorTok}[1]{\textcolor[rgb]{0.56,0.35,0.01}{\textit{#1}}}
\newcommand{\RegionMarkerTok}[1]{#1}
\newcommand{\SpecialCharTok}[1]{\textcolor[rgb]{0.00,0.00,0.00}{#1}}
\newcommand{\SpecialStringTok}[1]{\textcolor[rgb]{0.31,0.60,0.02}{#1}}
\newcommand{\StringTok}[1]{\textcolor[rgb]{0.31,0.60,0.02}{#1}}
\newcommand{\VariableTok}[1]{\textcolor[rgb]{0.00,0.00,0.00}{#1}}
\newcommand{\VerbatimStringTok}[1]{\textcolor[rgb]{0.31,0.60,0.02}{#1}}
\newcommand{\WarningTok}[1]{\textcolor[rgb]{0.56,0.35,0.01}{\textbf{\textit{#1}}}}
\usepackage{graphicx}
\makeatletter
\def\maxwidth{\ifdim\Gin@nat@width>\linewidth\linewidth\else\Gin@nat@width\fi}
\def\maxheight{\ifdim\Gin@nat@height>\textheight\textheight\else\Gin@nat@height\fi}
\makeatother
% Scale images if necessary, so that they will not overflow the page
% margins by default, and it is still possible to overwrite the defaults
% using explicit options in \includegraphics[width, height, ...]{}
\setkeys{Gin}{width=\maxwidth,height=\maxheight,keepaspectratio}
% Set default figure placement to htbp
\makeatletter
\def\fps@figure{htbp}
\makeatother
\setlength{\emergencystretch}{3em} % prevent overfull lines
\providecommand{\tightlist}{%
  \setlength{\itemsep}{0pt}\setlength{\parskip}{0pt}}
\setcounter{secnumdepth}{-\maxdimen} % remove section numbering
\ifLuaTeX
  \usepackage{selnolig}  % disable illegal ligatures
\fi
\IfFileExists{bookmark.sty}{\usepackage{bookmark}}{\usepackage{hyperref}}
\IfFileExists{xurl.sty}{\usepackage{xurl}}{} % add URL line breaks if available
\urlstyle{same} % disable monospaced font for URLs
\hypersetup{
  pdftitle={Tarea25},
  pdfauthor={Yimmy Eman},
  hidelinks,
  pdfcreator={LaTeX via pandoc}}

\title{Tarea25}
\author{Yimmy Eman}
\date{2022-07-17}

\begin{document}
\maketitle

\hypertarget{pregunta-1}{%
\section{Pregunta 1}\label{pregunta-1}}

Haz un dibujo de las siguientes listas anidadas:

• list(a, b, list(c, d), list(e, f)) •
list(list(list(list(list(list(a))))))

\hypertarget{pregunta-2}{%
\section{Pregunta 2}\label{pregunta-2}}

¿Qué pasa si haces un subconjunto de una tibble con la sintaxis de cómo
si se tratara de una lista? Infiere entonces el comportamiento que
permite diferenciar una tibble de una lista.

Subdividir un tibble funciona de la misma manera que una lista; un marco
de datos se puede considerar como una lista de columnas. La diferencia
clave entre una lista y un tibble es que todos los elementos (columnas)
de un tibble deben tener la misma longitud (número de filas). Las listas
pueden tener vectores con diferentes longitudes como elementos.

\hypertarget{pregunta-3}{%
\section{Pregunta 3}\label{pregunta-3}}

¿Qué devuelve la función hms::hms(3600) y cómo lo imprime? Investiga
sobre qué tipo de dato primitivo está construido este vector aumentado y
qué atributos utiliza.

\begin{Shaded}
\begin{Highlighting}[]
\NormalTok{x }\OtherTok{\textless{}{-}}\NormalTok{ hms}\SpecialCharTok{::}\FunctionTok{hms}\NormalTok{(}\DecValTok{3600}\NormalTok{)}
\FunctionTok{class}\NormalTok{(x)}
\end{Highlighting}
\end{Shaded}

\begin{verbatim}
## [1] "hms"      "difftime"
\end{verbatim}

\begin{Shaded}
\begin{Highlighting}[]
\NormalTok{x}
\end{Highlighting}
\end{Shaded}

\begin{verbatim}
## 01:00:00
\end{verbatim}

\hypertarget{pregunta-4}{%
\section{Pregunta 4}\label{pregunta-4}}

Intenta crear una tibble con columnas que tengan diferente longitud.
¿Qué ocurre?

\begin{Shaded}
\begin{Highlighting}[]
\CommentTok{\#tibble(x = 1, y = 1:5)}
\end{Highlighting}
\end{Shaded}

\begin{Shaded}
\begin{Highlighting}[]
\CommentTok{\#tibble(x = 1:3, y = 1:4)}
\CommentTok{\# Error:}
\CommentTok{\#! Tibble columns must have compatible sizes.}
\CommentTok{\# Size 3: Existing data.}
\CommentTok{\# Size 4: Column \textasciigrave{}y\textasciigrave{}.}
\CommentTok{\#ℹ Only values of size one are recycled.}
\CommentTok{\#Backtrace:}
\CommentTok{\# 1. tibble::tibble(x = 1:3, y = 1:4)}
\CommentTok{\# 2. tibble:::tibble\_quos(xs, .rows, .name\_repair)}
\CommentTok{\# 3. tibble:::vectbl\_recycle\_rows(res, first\_size, j, given\_col\_names[[j]])}
\end{Highlighting}
\end{Shaded}

\hypertarget{pregunta-5}{%
\section{Pregunta 5}\label{pregunta-5}}

Según lo que has practicado en esta tarea, ¿está bien tener una lista
como columna de una tibble? ¿Por qué?

El mensaje de error se refería a vectores que tenían diferentes
longitudes. Pero no hay nada que impida que un tibble tenga vectores de
diferentes tipos: dobles, carácter, enteros, lógico, factor, fecha. Los
últimos siguen siendo atómicos, pero tienen atributos adicionales.
Entonces, tal vez no haya problemas con un vector de lista siempre que
tenga la misma longitud.

\end{document}
